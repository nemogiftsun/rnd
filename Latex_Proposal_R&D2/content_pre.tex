
\section{Introduction}
 \paragraph{}
 Manipulation with a mobile base has attracted a lot of research in the last decade because of various opportunities for applications in service and industrial robotics. These 
 systems combine both the kinematics of the arm and base providing redundant degrees of freedom to increase the workspace of the manipulator. Open loop control strategies to execute manipulation tasks are limited in their applications as they suffer from imprecision and lack of robustness due to noise in sensors and actuators. The perceived information keeps changing with the environment bringing the necessity to have a control that tightly couples the perception information and the manipulation actions of the robot. This allows the robot perform precise and reliable manipulation of objects. Moreover, mobile manipulators in general are supposed to exploit the kinematics and dynamics of both the arm and base for executing complex tasks. But arm and base navigation components are most often seen as separate entities when it comes to robot control which makes it difficult to coordinate for an advanced task like manipulating an object on a moving platform complying to both the kinematic and goal constraints. 
 
 This work is focused on achieving a robust and dynamic whole body control in Mobile Manipulators which is generic enough to plug in sensory feedback directly and execute planned motions using an environment model constructed using the perception components. This task will concentrate on developing 'Stack of Tasks' controller, a heirarchical task
 based velocity controller for KUKA youBot to perform reliable manipulation in an industrial setup.
 \section{Motivation}
 \paragraph{}
 Industrial scenarios in general might require a mobile robot to handle from simple to complex manipulation tasks. A simple task can be to pick and place static industrial objects in a pre-defined workspace which has some challenges like detecting these objects, identify feasible grasping points, plan and control arm actions and avoid obstacles during execution. This task can be executed by placing the base optimally with respect to the object and use an arm controller to manipulate. The control demand for successful execution is not so complex provided the perception components are precise enough. But the object pose information from the perception pipeline may suffer from errors due to calibration, sensor noise, and photometric effects. A visual servo control can be used in such cases to make the control loop closed to achieve successful execution. A standard technique uses classical task function based control which minimizes the error between the goal and current configuration. This control can be either 
image based or position based which works differently at different situations demanding experimental analysis to choose the right one for the scenario. It is quite overwhelming that it demands a lot of effort in developing a reliable control for some tasks in an Industrial scenario. 
 
  \paragraph{}
  A complex task can be manipulating a moving object on a conveyor belt or exchanging objects between multiple robots. These tasks are much more challenging as it requires a precise object tracking system and a closed loop control that can synchronize arm and base actions. In some situations, it might even require multiple robots to perform a cooperative manipulation task which makes the scenario much more complex. The scenarios point out that each task has its own complexity in terms of the demand it requires from the perception components and the robot controller. This makes it complex to develop reliable applications for mobile manipulators in an industrial setup. This motivates a necessity for a generic task based controller which is flexible enough to be used in developing real time mobile robot applications. 

 \section{Problem Formulation}
\paragraph{}
  Mobile Manipulators are usually redundant systems due to the kinematics of both the base and arm. Control of such redundant robots is difficult as it is not always easy to compute analytical inverse kinematic and dynamic solutions. The control techniques to solve redundancy mostly uses task function based approach with instantaneous inverse jacobian mapping of the robot to compute optimal controller outputs. Generalized Inverse Kinematics(GIK) by Hanafusa et al. uses a damped least squares to optimize the control of redundant degrees of freedom which is quite popular among the community working on Humanoid robots\cite{hanafusa1981analysis}. Redundant manipulators provide a lot of functional advantages in terms of successful task execution while having extra degrees of freedom to handle collision avoidance, singularity avoidance, obstacle avoidance and various other constraints. A framework to plug in various constraints on the generated solutions was proposed by Siciliano et al. \cite{siciliano1991general}.
 He extended Hanafusa's approach using an iterative scheme to handle multiple tasks for highly redundant robots. But Siciliano's framework does not take inequality constraints into account which is quite frequently encountered in real time applications. 
  
\paragraph{}  
 Some solutions were focused on converting inequality constraints to equality constraints in a systemetic way to plug in the task priority framework. One such solution was plugging in a log barrier function equivalent to the corresponding inequality constraint which can be a joint limit avoidance or obstacle avoidance constraint\cite{khatib1986real}. But these constraints influences the system even far away from the joint limits or obstacles when they are assigned high priority. A lot of solutions to plug inequality constraints were proposed.But unfortunately these approaches in general have priority inversion problems which makes it unreliable and discontinuous \cite{chan1995weighted} \cite{mansard2009directional} \cite{raunhardt2007progressive}. Mansard et al. suggested the idea of realizing homotopy between control laws with and without avoidance constraints in which balance in homotopy factors ensures priorities in objectives to be taken into account \cite{Mansard2009}. But this approach is quite expensive and the cost reduction by Lee et al. \cite{lee2012intermediate} necessitates study for each new task to be plugged in the framework.  

\paragraph{}  
The alternative solutions encode inequality constraints as a least-square program. A cascade approach to solve quadratic optimization problems heirarchically was first presented by Kanoun et al for redundant manipulators\cite{kanoun2011kinematic}. De Lasa et al. improved the computational cost of the solver but allows encoding inequality constraints as higher priority tasks \cite{de2010feature}. Escande et al worked on the linear decomposition part to reduce repeated computations \cite{escande2010fast}. A generic framework which can take both equality and inequality constraints at any priority level has been recently proposed. This approach uses an active search algorithm instead of cascade approach used previously. The control scheme has been validated to be stable and continuous for a given heirarchy of tasks on a redundant robot for an inverse kinematic problem with obstacle avoidance constraints. Continuity of the control scheme is not ensured when tasks are updated or removed during the runtime. 

\paragraph{}
It is questionable whether these approaches are robust enough to be used in real time mobile robot applications.
In Industrial scenarios, we need a control scheme that 
\begin{itemize}
\item  is time efficient
\item  can handle inequality and equality constraints at any priority level  
\item  switch tasks dynamically without any control discontinuity
\item  generic enough to handle both motion generation and path following
\end{itemize}

 
